\subsection{Role CPD}

The goal of RoleCPD \cite{Kim2022SoccerCPD} is to detect long-term tactical changes in player roles (e.g., a winger swapping sides with another winger permanently) while ignoring temporary switches (e.g., overlapping runs or covering defensive duties). This process operates within a single Formation Period $(T_i)$ identified in the previous step.

\subsubsection{Formal Representation}
We define the inputs and mathematiccal framework based on the SoccerCPD protocol:
\begin{itemize}
  \item \textbf{Input:} A sequence of "Temporary Role Permutations" $\{\pi_t\}_{t=1}^{|T_i|}$.
  \item \textbf{Role Permutation ($\pi_t$):} At every frame $t$, the Role Representation step assigns a role $X_p$ to every player $p$. Since roles are distinct, this assignement is a permutation of the initial canonical role: $$\beta_t (p) = \pi_t (X_p)$$ Where $\beta_t$ is the player-to-temporary-role mapping at time $t$.
\end{itemize}

\subsubsection{The Distance Metric}
To detect a change, we must quantify the difference between the team's configuration at time $t$ and time $t'$. Since the data consists of permutations (non-Euclidean), we cannot use standard Euclidean distance. We use the Hamming Distance normalized by the number of roles ($N = 10$ outfield players): $$d(\pi_t, \pi_{t'}) = \frac{1}{N} \sum_{p \in P} \mathbbm{1}_{\pi_t(X_p) \neq \pi_{t'}(X_p)}$$ This metric represents the "Switch Rate" or the proportion of players whose roles differ between two frames.

\subsubsection{Change-Point Detection Algorithm}
The paper utilizes Discrete g-segmentation, a graph-based change-point detection method effective for repeated observations in non-Euclidean space.
The procedure is as follows:
\begin{enumerate}
  \item \textbf{Preprocessing:} Calculate the Switch Rate relative to the dominant permutation. Exclude frames with a switch rate $> 0.7$ (likely temporary switches during set-pieces or abnormal situations).
  \item \textbf{Segmentation:} Apply the dectection algorithm recursively on the sequence of permutations using the Hamming distance.
  \item \textbf{Significance Test:} A change point $\tau$ is significant if:
    \begin{itemize}
      \item The p-value of the scan statistic is $< 0.01$.
      \item The segmentation duration is sufficient (robustness against noise).
      \item The most frequent permutations (Instructed Roles) in the segments before and after $\tau$ must be distinct.
    \end{itemize}
\end{enumerate}

