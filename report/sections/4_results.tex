\section{Results} \label{sec:results}

\paragraph{Comparison across CPD backends (single match).}
We compare several formation CPD backends (\texttt{gseg\_avg}, \texttt{gseg\_union}, \texttt{kernel\_rbf}, \texttt{kernel\_linear}) on a single reference match, using identical preprocessing and hyperparameters. The goal is not to rank methods, but to assess whether different CPD formulations yield compatible formation change-point timelines.

Figure \ref{fig:exp1_cp_timeline_methods} shows the detected formation change-points for each backend on a common timeline. Despite differences in underlying detection principles, most methods identify change-points at similar temporal locations, indicating a shared high-level segmentation of the match. Minor discrepancies mainly correspond to additional or missing change-points rather than large temporal shifts, suggesting differences in sensitivity rather than fundamentally different interpretations of the match dynamics.

Overall, this experiment indicates that formation change-points are largely consistent across CPD backends for this match. Since the analysis is restricted to a single game, these observations should be interpreted qualitatively and primarily serve to illustrate methodological robustness rather than general performance.

\begin{figure}[t]
    \centering
    \includegraphics[width=0.9\linewidth]{../figures/CPD_backends.png}
    \caption{
    Formation change-points detected by different CPD backends on a common timeline.
    }
    \label{fig:exp1_cp_timeline_methods}
\end{figure}


\subsection{Experiments using SoccerCPD}

\paragraph{Sensitivity analysis of formation CPD (single match).}
We evaluate the sensitivity of the formation change-point detection (CPD) pipeline to its main hyperparameters on a single reference match. All experiments use the same CPD backend (\texttt{gseg\_avg}), and one parameter is varied at a time while keeping the others fixed.

We first study the impact of the formation distance threshold \(\texttt{min\_fdist}\), which controls the minimum Manhattan distance between mean formation graphs required to validate a change-point. As \(\texttt{min\_fdist}\) increases, the number of detected formation segments decreases monotonically, indicating that higher thresholds progressively filter out minor structural variations. Importantly, the remaining change-points occur at consistent temporal locations, suggesting that \(\texttt{min\_fdist}\) mainly acts as a pruning mechanism rather than inducing temporal instability.

We then analyze sensitivity to the minimum period duration \(\texttt{min\_pdur}\), which enforces a lower bound on segment length. Over the tested range, both the number and locations of detected change-points remain unchanged, indicating that the formation changes identified in this match are temporally well separated and not driven by short-lived fluctuations.

Overall, these results suggest that, for this match, formation change-points are robust to reasonable variations in temporal resolution and structural significance thresholds. Since the analysis is conducted on a single match, these findings should be interpreted as illustrative rather than representative, motivating a broader multi-match evaluation.

\begin{figure}[t]
    \centering
    \begin{subfigure}[t]{0.42\linewidth}
        \centering
        \includegraphics[width=\linewidth]{../figures/min_fdist.png}
        \caption{Number of formation segments}
        \label{fig:exp2_segments_fdist}
    \end{subfigure}
    \hfill
    \begin{subfigure}[t]{0.55\linewidth}
        \centering
        \includegraphics[width=\linewidth]{../figures/exp2_cp_timeline_min_fdist.png}
        \caption{Formation change-point timeline}
        \label{fig:exp2_cp_timeline_fdist}
    \end{subfigure}
    \caption{
    Sensitivity of formation change-point detection to the formation distance threshold \texttt{min\_fdist}.
    Increasing the threshold reduces the number of detected segments while preserving the temporal locations of the most salient change-points.
    }
    \label{fig:exp2_fdist}
\end{figure}



\subsection{Extensions to SoccerCPD} \label{sec:extensions}
To enhance the interpretability of the tactical segments identified by SoccerCPD, we introduced two novel analytical layers: a contextual separation of formations based on possession and a quantifying metric for role fluidity.

\subsubsection{Possession Context: Attacking vs. Defending Shapes}
A limitation of the standard FormCPD output is that it produces a single "average" formation for a detected time segment. However, modern soccer teams adopt vastly different shapes depending on possession status.
We extended the pipeline to filter frames based on the possession context (Attack vs. Defense, based on a centroid threshold since possession event data is unavailable for GPS tracking) before computing the formation centroids. 
\begin{itemize}
    \item \textbf{Attacking Formation:} Computed using only frames where the team is in possession. 
    \item \textbf{Defensive Formation:} Computed using frames where the opponent is in possession.
\end{itemize}
The results, visualized in Figure~\ref{fig:exp_att_def_formations} (Appendix~\ref{sec:annex_results}) for Match ID 1925299 (A-League, SkillCorner) and Figure~\ref{fig:exp_barca_formations_by_phase} (FCBarcelona, StatsBomb), reveal significant structural differences that are lost in a global average. For both matches, the attacking shape is wider and higher up the pitch, whereas the defensive shape compresses, with wingers dropping deeper to form a compact block. This granular analysis confirms that a single "stable" phase detected by SoccerCPD actually comprises two distinct, alternating tactical configurations. The StatsBomb FCBarcelona match illustrates this particularly well, with the team shifting from a high-pressing 4-3-3 when attacking to a more conservative 4-4-2 block when defending. It must be noted, however, that the sparsity of event-stream data limits the precision of these formation estimates compared to high-frequency tracking data. This is why we see a player such as Pjanic, who typically plays as a central midfielder, being detected in a very deep defensive position in both the attacking and defensive formations of Phase 1. The events involving him mostly occur when the team is defending, leading to a biased estimation of his average position. That is not the case in phase 2, where he is detected in a more reasonable midfield position.

\subsubsection{Player Stationarity Metric}
While the original SoccerCPD model assigns a Gaussian mean ($\mu_k$) and covariance ($\Sigma_k$) to each role, it does not explicitly quantify how strictly a player adheres to that role. We defined a Player Stationarity metric, calculated as the standard deviation of the Euclidean distance between a player's actual position and their assigned role's center ($\mu_k$) over a formation phase.
A lower value indicates a "fixed" positional role (e.g., a center-back holding the line), while a higher value indicates a "fluid" or "roaming" role. Applying this to the sampled StatsBomb FC Barcelona match, we observed distinct behavioral patterns. As shown in Figure~\ref{fig:exp_player_stationarity_phase1_barca} in Appendix~\ref{sec:annex_results}, midfield roles exhibited higher stationarity (lower deviation) compared to defensive fullbacks or attacking roles. For instance, during Phase 1, the average spatial deviation was approximately 9.8m, but specific attacking and even defensive roles showed deviations exceeding 12m, quantifying the "free-roaming" nature of Barcelona players compared to their midfielders, who distribute the ball and keep possession. These results can also be compared to phase 2 in Figure~\ref{fig:exp_player_stationarity_phase2_barca}, where the average spatial deviation increased, as players where much more fluid in both attack and defense.

We also validated this metric on the A-League tracking data (Match ID 1925299), as illustrated in Figure~\ref{fig:exp_player_stationarity_phase1}. The metric successfully captures the variance in player movement within a stable tactical phase, providing coaches with a concrete measure of positional discipline.

\section{Conclusion}
This report presented a comprehensive reproduction and extension of the SoccerCPD framework for detecting tactical formation changes in soccer using spatiotemporal data. We successfully implemented both stages of the pipeline, FormCPD and RoleCPD, verifying their effectiveness on synthetic datasets and real-world matches. Our contributions included the introduction of a player stationarity metric to quantify role fluidity, as well as the refinement of formation analysis by separating frames based on possession context (attack vs. defense). These extensions provided deeper insights into team tactics and individual player behaviors. We finally applied the adapted framework to event-stream data from the StatsBomb repository, demonstrating its versatility across different data types. Overall, our work validates the SoccerCPD approach while enhancing its interpretability and applicability in football analytics.