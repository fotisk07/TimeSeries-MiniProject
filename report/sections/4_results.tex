\section{Results}

\paragraph{Comparison across CPD backends (single match).}
We compare several formation CPD backends (\texttt{gseg\_avg}, \texttt{gseg\_union}, \texttt{kernel\_rbf}, \texttt{kernel\_linear}) on a single reference match, using identical preprocessing and hyperparameters. The goal is not to rank methods, but to assess whether different CPD formulations yield compatible formation change-point timelines.

Figure \ref{fig:exp1_cp_timeline_methods} shows the detected formation change-points for each backend on a common timeline. Despite differences in underlying detection principles, most methods identify change-points at similar temporal locations, indicating a shared high-level segmentation of the match. Minor discrepancies mainly correspond to additional or missing change-points rather than large temporal shifts, suggesting differences in sensitivity rather than fundamentally different interpretations of the match dynamics.

Overall, this experiment indicates that formation change-points are largely consistent across CPD backends for this match. Since the analysis is restricted to a single game, these observations should be interpreted qualitatively and primarily serve to illustrate methodological robustness rather than general performance.

\begin{figure}[t]
    \centering
    \includegraphics[width=0.9\linewidth]{../figures/CPD_backends.png}
    \caption{
    Formation change-points detected by different CPD backends on a common timeline.
    }
    \label{fig:exp1_cp_timeline_methods}
\end{figure}


\subsection{Experiments using SoccerCPD}

\paragraph{Sensitivity analysis of formation CPD (single match).}
We evaluate the sensitivity of the formation change-point detection (CPD) pipeline to its main hyperparameters on a single reference match. All experiments use the same CPD backend (\texttt{gseg\_avg}), and one parameter is varied at a time while keeping the others fixed.

We first study the impact of the formation distance threshold \(\texttt{min\_fdist}\), which controls the minimum Manhattan distance between mean formation graphs required to validate a change-point. As \(\texttt{min\_fdist}\) increases, the number of detected formation segments decreases monotonically, indicating that higher thresholds progressively filter out minor structural variations. Importantly, the remaining change-points occur at consistent temporal locations, suggesting that \(\texttt{min\_fdist}\) mainly acts as a pruning mechanism rather than inducing temporal instability.

We then analyze sensitivity to the minimum period duration \(\texttt{min\_pdur}\), which enforces a lower bound on segment length. Over the tested range, both the number and locations of detected change-points remain unchanged, indicating that the formation changes identified in this match are temporally well separated and not driven by short-lived fluctuations.

Overall, these results suggest that, for this match, formation change-points are robust to reasonable variations in temporal resolution and structural significance thresholds. Since the analysis is conducted on a single match, these findings should be interpreted as illustrative rather than representative, motivating a broader multi-match evaluation.

\begin{figure}[t]
    \centering
    \begin{subfigure}[t]{0.42\linewidth}
        \centering
        \includegraphics[width=\linewidth]{../figures/min_fdist.png}
        \caption{Number of formation segments}
        \label{fig:exp2_segments_fdist}
    \end{subfigure}
    \hfill
    \begin{subfigure}[t]{0.55\linewidth}
        \centering
        \includegraphics[width=\linewidth]{../figures/exp2_cp_timeline_min_fdist.png}
        \caption{Formation change-point timeline}
        \label{fig:exp2_cp_timeline_fdist}
    \end{subfigure}
    \caption{
    Sensitivity of formation change-point detection to the formation distance threshold \texttt{min\_fdist}.
    Increasing the threshold reduces the number of detected segments while preserving the temporal locations of the most salient change-points.
    }
    \label{fig:exp2_fdist}
\end{figure}
