\documentclass[11pt]{article}
\usepackage{theme}
\usepackage{subcaption}
\usepackage{shortcuts}
% Document parameters
% Document title
\title{Mini-Project (ML for Time Series) - MVA 2025/2026}

\author{
Adonis Jamal \email{adonis.jamal@student-cs.fr} \\ % student 1
Fotios Kapotos \email{fotiskapotos@gmail.com} % student 2
}

\begin{document}
\maketitle

\paragraph{What is expected for these mini-projects?}
The goal of the exercise is to read (and understand) a research article, implement it (or find an implementation), test it on real data and comment on the results obtained.
Depending on the articles, the task will not always be the same: some articles are more theoretical or complex, others are in the direct line of the course, etc... It is therefore important to balance the exercise according to the article. For example, if you have reused an existing implementation, it is obvious that you will have to develop in a more detailed way the analysis of the results, the influence of the parameters etc... Do not hesitate to contact us by email if you wish to be guided.

\paragraph{The report}
 The report must be at most FIVE pages and use this template (excluding references). If needed, additional images and tables can be put in Appendix, but must be discussed in the main document. The report must contain a precise description of the work done, a description of the method, and the results of your tests. Please do not include source code! The report must clearly show the elements that you have done yourself and those that you have reused only, as well as the distribution of tasks within the team (see detailed plan below.)
 
 \paragraph{The source code}
In addition to this report, you will have to send us a Python notebook allowing to launch the code and to test it on data. For the data, you can find it on standard sites like Kaggle, or the site https://timeseriesclassification.com/ which contains a lot of signals!


\paragraph{The oral presentations}
They will last 10 minutes followed by 5 minutes of questions. The plan of the defense is the same as the one of the report: presentation of the work done, description of the method and analysis of the results.


\paragraph{Deadlines}
Two sessions will be available :
\begin{itemize}
 \item \textbf{Session 1}
 \begin{itemize}
  \item Deadline for report: December 14th (23:59)
  \item Oral presentations: December 15th and 17th (precise times TBA)
 \end{itemize}
\item \textbf{Session 2}
 \begin{itemize}
  \item Deadline for report: January 4th (23:59)
  \item Oral presentations: January, 5th and 7th (precise times TBA)
 \end{itemize}
\end{itemize}

\section{Introduction and contributions}

The Introduction section (indicative length : less than 1 page) should detail the scientific context of the article you chose, as well as the task that you want to solve (especially if you apply it on novel data). \textbf{The last paragraph of the introduction must contain the following information}:
\begin{itemize}
    \item Repartition of work between the two students
    \item Use of available source code or not, percentage of the source code that has been reused, etc.
    \item Use of existing experiments or new experiments (e.g. test of the influence of parameter that was not conducted in the original article, application of the method on a novel task/data set etc.)
    \item Improvement on the original method (e.g. new pre/post processing steps, grid search for optimal parameters etc.)
\end{itemize}

\section{Methodology}
\subsection{Formation Change-Point Detection (FormCPD)}

This subsection presents the full pipeline used to detect tactical formation changes (FormCPD), introduced in \cite{Kim2022SoccerCPD}. The objective is to start from a sequence of feature matrices $\{A(t)\}_{t \in T}$ and infer a partition of the timeline into segments $T_1 < \cdots < T_m$ such that, for all $t \in T_i$, the features $A(t)$ follow a common underlying distribution $\mathcal{F}_i$.

Our presentation proceeds in four steps:
\begin{enumerate}
  \item estimation of consistent player roles using a constrained expectation--maximization algorithm combined with frame-wise Hungarian matching (Section~4.1.1),
  \item construction of role-adjacency matrices via Delaunay triangulation of the role positions (Section~4.1.2),
  \item detection of formation change-points on the resulting graph sequence using discrete g-segmentation with a Manhattan distance metric (Section~4.2),
  \item clustering of segmented formations through role alignment and hierarchical aggregation to extract canonical formation types (Section~4.3).
\end{enumerate}

\subsubsection{Role Assignment via Constrained EM}

The first stage of FormCPD consists in assigning each player a \emph{latent spatial role} at every frame, following the representation proposed by Bialkowski et al.~\cite{bialkowski2014}. The goal is to model persistent spatial zones rather than player identities, as players may temporarily exchange positions while the global formation remains unchanged.

Each role \( k \in \{1,\dots,N\} \) is modeled as a Gaussian component with parameters \( (\mu_k, \Sigma_k) \). At time \( t \), an observed player position \( x_{i,t} \in \mathbb{R}^2 \) is assumed to follow
\[
x_{i,t} \sim \mathcal{N}(\mu_{z_{i,t}}, \Sigma_{z_{i,t}}),
\]
with the hard constraint that no two players may occupy the same role at a given frame. Estimation is performed using an EM procedure. During the E-step, a cost matrix is built from negative log-likelihoods and a Hungarian assignment enforces a one-to-one mapping between players and roles at each frame. During the M-step, role means and covariances are updated using all positions assigned to each role over the session.

Unlike standard Gaussian mixture models, responsibilities are both \emph{hard} and \emph{frame-wise coupled} through Hungarian matching, while temporal consistency is implicitly enforced by sharing role parameters across frames.

Figure~\ref{fig:role_swap} illustrates the effect of this representation on a simulated role-swapping scenario. When trajectories are colored by player identity, crossings obscure any stable spatial structure. Recoloring the same data by estimated roles instead reveals coherent clusters: players exchange colors as they change positions, while role labels remain anchored to fixed spatial zones. This demonstrates the key motivation of role-based modeling. However, the approach assumes that spatial roles are sufficiently separable; when clusters overlap or positional variance becomes large, role estimation may become unstable.

\begin{figure}
    \includegraphics[width=1\linewidth]{figures/role_swap.png}
    \caption{Player versus role-based coloring of trajectories during a simulated position swap.}
    \label{fig:role_swap}
\end{figure}

\subsubsection{Encoding Spatial Structure via Delaunay Triangulation}

Once role assignments are available, each frame is encoded as a graph describing the local spatial relationships between roles. This is obtained by computing a Delaunay triangulation of the role locations
\[
V(t) = \{v_1(t), \dots, v_N(t)\} \subset \mathbb{R}^2 .
\]

The Delaunay triangulation connects two roles when they share an edge in the triangulated mesh, defining adjacency based solely on relative geometry without introducing distance thresholds or fixed neighborhood sizes. From this, a binary role-adjacency matrix
\[
A(t) \in \{0,1\}^{N \times N}
\]
is constructed as
\[
A_{k\ell}(t) =
\begin{cases}
1, & \text{if roles } v_k(t) \text{ and } v_\ell(t) \text{ are adjacent,} \\
0, & \text{otherwise.}
\end{cases}
\]

Each frame is therefore represented as an undirected graph whose vertices correspond to roles and whose edges encode local spatial proximity. This representation emphasizes the \emph{topological} structure of the formation rather than precise metric distances. Combined with role indexing, it remains invariant to player permutations: transient exchanges of positions do not alter the role-adjacency graph as long as players remain within the same spatial zones.

Figure~\ref{fig:delaunay_examples} shows Delaunay triangulations computed from simulated role positions before and after a formation change from 4--4--2 to 3--4--3. The change induces a clear reconfiguration of spatial adjacency relationships.

Figure~\ref{fig:adjacency_experiment} presents the corresponding role-adjacency matrices. Instantaneous matrices (top row) exhibit small fluctuations due to geometric noise, whereas averaging over each formation segment (bottom row) yields stable, formation-specific patterns used as inputs for clustering.

\begin{figure}[t]
    \centering
    \includegraphics[width=0.9\linewidth]{figures/delany_form_change.png}
    \caption{Delaunay triangulations of role locations at two frames before and after a synthetic formation change (4--4--2 $\rightarrow$ 3--4--3). Nodes represent roles and edges represent spatial adjacencies.}
    \label{fig:delaunay_examples}
\end{figure}

\begin{figure}[t]
  \centering
    \begin{subfigure}[t]{0.48\linewidth}
        \centering
        \includegraphics[height=3.8cm,keepaspectratio]{figures/A_examples.png}
        \caption{Single-frame binary adjacency matrices before and after the simulated formation change (4--4--2 $\rightarrow$ 3--4--3).}
        \label{fig:adj_frames}
    \end{subfigure}
    \hfill
    \begin{subfigure}[t]{0.48\linewidth}
        \centering
        \includegraphics[height=3.8cm,keepaspectratio]{figures/A_mean.png}
        \caption{Mean role-adjacency matrices averaged over each formation period}
        \label{fig:adj_means}
    \end{subfigure}
    \caption{Delaunay-based adjacency representation on a synthetic formation change.}
    \label{fig:adjacency_experiment}
\end{figure}



\subsection{Role Change-Point Detection (RoleCPD)}
The goal of RoleCPD \cite{Kim2022SoccerCPD} is to detect long-term tactical changes in player roles (e.g., a winger swapping sides with another winger permanently) while ignoring temporary switches (e.g., overlapping runs or covering defensive duties). This process operates within a single Formation Period $(T_i)$ identified in the previous step.

\subsubsection{Formal Representation}
We define the inputs and mathematiccal framework based on the SoccerCPD protocol:
\begin{itemize}
  \item \textbf{Input:} A sequence of "Temporary Role Permutations" $\{\pi_t\}_{t=1}^{|T_i|}$.
  \item \textbf{Role Permutation ($\pi_t$):} At every frame $t$, the Role Representation step assigns a role $X_p$ to every player $p$. Since roles are distinct, this assignement is a permutation of the initial canonical role: $$\beta_t (p) = \pi_t (X_p)$$ Where $\beta_t$ is the player-to-temporary-role mapping at time $t$.
\end{itemize}

\subsubsection{The Distance Metric}
To detect a change, we must quantify the difference between the team's configuration at time $t$ and time $t'$. Since the data consists of permutations (non-Euclidean), we cannot use standard Euclidean distance. We use the Hamming Distance normalized by the number of roles ($N = 10$ outfield players): $$d(\pi_t, \pi_{t'}) = \frac{1}{N} \sum_{p \in P} \mathbbm{1}_{\pi_t(X_p) \neq \pi_{t'}(X_p)}$$ This metric represents the "Switch Rate" or the proportion of players whose roles differ between two frames.

\subsubsection{Change-Point Detection Algorithm}
The paper utilizes Discrete g-segmentation, a graph-based change-point detection method effective for repeated observations in non-Euclidean space.
The procedure is as follows:
\begin{enumerate}
  \item \textbf{Preprocessing:} Calculate the Switch Rate relative to the dominant permutation. Exclude frames with a switch rate $> 0.7$ (likely temporary switches during set-pieces or abnormal situations).
  \item \textbf{Segmentation:} Apply the dectection algorithm recursively on the sequence of permutations using the Hamming distance.
  \item \textbf{Significance Test:} A change point $\tau$ is significant if:
    \begin{itemize}
      \item The p-value of the scan statistic is $< 0.01$.
      \item The segmentation duration is sufficient (robustness against noise).
      \item The most frequent permutations (Instructed Roles) in the segments before and after $\tau$ must be distinct.
    \end{itemize}
\end{enumerate}



\section{Data}
The Data section (indicative length : 1 page) should provide a deep analysis of the data used for experiment. In particular, we are interested here in your capacity to provide relevant and thoughtful feedbacks on the data and to demonstrate that you master some "data diagnosis" tools that have been dealt with in the lectures/tutorials.

\section{Results}
The Result section (indicative length : 1 to 2 pages) \textbf{should display numerical simulations on real time series} (even if the original article was focused on images). If you re-used some existing implementations, it is expected that this section develops new experiments that were not present in the original article. Results should be discussed not only based on quantative scores but also on qualitative aspects. In particular (especially if your article focuses on black box methods), please provide some feedbacks whether the method was adapted to the data or not and whether the hypothesis behind the approach you used were validated or not.




\bibliographystyle{plain}
\bibliography{bib}

\end{document}

