\section{Introduction}
\subsection{Context and Problem Statement}
\begin{frame}{Context}
    \frametitle{Context and Problem Statement}
    \begin{itemize}
        \item \textbf{Context:} Analyzing team formation is crucial for interpreting tactics in fluid sports like soccer.
        \item \textbf{The Challenge:} Tracking data is noisy; players frequently switch positions temporarily.
        \item \textbf{Limitation of existing methods:} Often assume constant formations or react too sensitively to frame-by-frame noise.
        \item \textbf{Objective:} Reproduce and extend \textbf{SoccerCPD} \cite{Kim2022SoccerCPD}, an unsupervised framework that distinguishes:
        \begin{itemize}
            \item \textbf{FormCPD:} Shifts in global spatial configuration.
            \item \textbf{RoleCPD:} Long-term tactical changes in individual roles.
        \end{itemize}
    \end{itemize}
\end{frame}

\subsection{Work Done \& Contributions}
\begin{frame}{Contributions}
    \frametitle{{Work Repartition and Contributions}}
    \begin{columns}
        \begin{column}{0.6\textwidth}
            \textbf{Work Repartition}
            \begin{itemize}
                \item \textbf{FormCPD:} Delaunay-based adjacency and formation CPD.
                \item \textbf{RoleCPD:} Synthetic role data, permutation analysis, recursive segmentation.
            \end{itemize}

            \vspace{0.3cm}

            \textbf{Implementation \& Extensions}
            \begin{itemize}
                \item Python package with R backend (g-segmentation).
                \item \textbf{Novel metric:} Player stationarity.
                \item \textbf{Refinement:} Possession-context formations.
            \end{itemize}
        \end{column}

        \begin{column}{0.4\textwidth}
            \centering
            \includegraphics[width=\linewidth]{../figures/role_swap.png}
            \captionof{figure}{Role vs Player coloring}
        \end{column}
    \end{columns}
\end{frame}
    
