\section{Results \& Extensions}

\begin{frame}{Experiments on SoccerCPD : Robustness Analysis}
\textbf{Comparison across CPD backends (single match).}
\begin{itemize}
    \item Four backends tested: \texttt{gseg\_avg}, \texttt{gseg\_union}, \texttt{kernel\_rbf}, \texttt{kernel\_linear}.
    \item Identical preprocessing and hyperparameters.
    \item All methods detect formation changes at similar times.
    \item Differences correspond to sensitivity (extra/missing CPs), not different segmentations.
\end{itemize}

\vspace{0.3cm}

\textbf{Sensitivity to hyperparameters (\texttt{gseg\_avg}).}
\begin{itemize}
    \item Increasing \texttt{min\_fdist} reduces the number of segments while preserving CP locations.
    \item \texttt{min\_fdist} acts as a pruning threshold for minor structural variations.
    \item Varying \texttt{min\_pdur} has no effect, indicating well-separated formation changes.
\end{itemize}

\end{frame}



\subsection{Extension 1: Possession Context}
\begin{frame}{Extension 1: Attack vs. Defense}
    \frametitle{Contextualizing Formations (GPS)}
    \textbf{Problem:} A single "average" centroid formation ignores game state.
    \vspace{0.2cm}
    
    \begin{columns}
        \begin{column}{0.4\textwidth}
            \textbf{Method:}
            \begin{itemize}
                \item Filter frames by possession status.
                \item Compute centroids separately.
            \end{itemize}
            \textbf{Insight:}
            \begin{itemize}
                \item \textcolor{blue}{Defense:} Compact block.
                \item \textcolor{red}{Attack:} Wide, expansive shape.
            \end{itemize}
        \end{column}
        \begin{column}{0.6\textwidth}
            \centering
            \includegraphics[width=\linewidth]{../figures/Experiments_AttDefFormations_1925299.png}
            \captionof{figure}{Match 1925299: Distinct shapes detected.}
        \end{column}
    \end{columns}
\end{frame}


\subsection{Extension 2: Player Stationarity}
\begin{frame}{Extension 2: Player Stationarity Metric}
    \frametitle{Quantifying Role Fluidity ($\sigma$)}
    
    \textbf{Definition:} Standard deviation of Euclidean distance from the role center.
    
    \begin{columns}
        \begin{column}{0.5\textwidth}
            \textbf{GPS Data Findings (A-League)}
            \begin{itemize}
                \item \textbf{Fixed Roles:} CBs show low $\sigma$ (Stationary).
                \item \textbf{Roaming Roles:} Wingers/Midfielders show high $\sigma$.
                \item Validates tactical roles beyond simple coordinates.
            \end{itemize}
        \end{column}
        \begin{column}{0.5\textwidth}
            \centering
            \includegraphics[width=\linewidth]{../figures/Experiments_Player_Stationarity_Phase1_1925299.png}
            \captionof{figure}{Stationarity profile (GPS Data).}
        \end{column}
    \end{columns}
\end{frame}


\subsection{Extension 3: Event Data Generalization}
\begin{frame}{Validation 1: Event Data Formations}
    \frametitle{Offensive vs. Defensive Structures (FC Barcelona)}
    
    We extended the "Possession Context" analysis to sparse StatsBomb data.
    
    \begin{columns}
        \begin{column}{0.5\textwidth}
            \textbf{Tactical Shapes in Sparse Data}
            \begin{itemize}
                \item Despite low data density, distinct shapes emerge when aggregated over time.
                \item \textbf{Defense:} High density in central zones (low width).
                \item \textbf{Attack:} Wingers push high and wide; full-backs overlap.
            \end{itemize}
        \end{column}
        \begin{column}{0.5\textwidth}
            \centering
            \includegraphics[width=\linewidth]{../figures/Experiments_Barca_Formations_By_Phase.png} 
            \captionof{figure}{Barcelona: Attack (Red) vs Defense (Blue).}
        \end{column}
    \end{columns}
\end{frame}


\begin{frame}{Validation 2: Event Data Stationarity}
    \frametitle{Quantifying Fluidity in Event Data}
    
    Does the "Stationarity Metric" hold up with sparse event streams?
    
    \begin{columns}
        \begin{column}{0.5\textwidth}
            \textbf{Comparison Results}
            \begin{itemize}
                \item \textbf{Consistency:} The "Fixed" vs. "Fluid" distinction persists.
                \item \textbf{Phase Analysis (Barcelona):} 
                \begin{itemize}
                    \item \textit{Attack Phase:} High deviation (Avg ~12m) $\to$ Fluidity.
                    \item \textit{Defense Phase:} Lower deviation (Avg ~9.6m) $\to$ Rigidity.
                \end{itemize}
            \end{itemize}
        \end{column}
        \begin{column}{0.5\textwidth}
            \centering
            \includegraphics[width=\linewidth]{../figures/Experiments_Barca_Stationarity_Phase1.png} 
            \captionof{figure}{Stationarity: FC Barcelona (Event Data).}
        \end{column}
    \end{columns}
\end{frame}