\section{Results \& Extensions}


%%%%%%%%%%%%%%%%%%%%%%%
% SENSITIVITY ANALYSIS
%%%%%%%%%%%%%%%%%%%%%%%
\subsection{Sensitivity Analysis}
\begin{frame}{Experiments on SoccerCPD : Robustness Analysis}
\textbf{Comparison across CPD backends (single match).}
\begin{itemize}
    \item Four backends tested: \texttt{gseg\_avg}, \texttt{gseg\_union}, \texttt{kernel\_rbf}, \texttt{kernel\_linear}.
    \item Identical preprocessing and hyperparameters.
    \item All methods detect formation changes at similar times.
    \item Differences correspond to sensitivity (extra/missing CPs), not different segmentations.
\end{itemize}

\vspace{0.3cm}

\textbf{Sensitivity to hyperparameters (\texttt{gseg\_avg}).}
\begin{itemize}
    \item Increasing \texttt{min\_fdist} reduces the number of segments while preserving CP locations.
    \item \texttt{min\_fdist} acts as a pruning threshold for minor structural variations.
    \item Varying \texttt{min\_pdur} has no effect, indicating well-separated formation changes.
\end{itemize}

\end{frame}



%%%%%%%%%%%%%%%%%%%%%%%
% FORMATIONS
%%%%%%%%%%%%%%%%%%%%%%%
\subsection{Extension 1: Possession Context}
\begin{frame}{Extension 1: Formations}
    \frametitle{Contextualizing Formations: Attack vs. Defense}
    \begin{block}{Problem}
        A single "average" centroid formation ignores game state.
    \end{block}
    \vspace{0.2cm}

    \begin{columns}
        \begin{column}{0.4\textwidth}
            \textbf{Method:}
            \begin{itemize}
                \item Filter frames by possession status.
                \item Compute formations separately.
            \end{itemize}
            \textbf{Insight:}
            \begin{itemize}
                \item \textcolor{blue}{Defense:} Compact block.
                \item \textcolor{red}{Attack:} New attacking shape.
            \end{itemize}
        \end{column}
        \begin{column}{0.6\textwidth}
            \centering
            \includegraphics[width=\linewidth]{../figures/Experiments_AttDefFormations_1925299.png}
            \captionof{figure}{Detected formations for match "1925299" (A-League, SkillCorner) separated by phase of play: defending (left) vs. attacking (right). }
        \end{column}
    \end{columns}
\end{frame}


%%%%%%%%%%%%%%%%%%%%%%%
% STATIONARITY
%%%%%%%%%%%%%%%%%%%%%%%
\subsection{Extension 2: Player Stationarity}
\begin{frame}{Extension 2: Player Stationarity Metric}
    \frametitle{Quantifying Role Fluidity via Stationarity}
    \begin{block}{Motivation}
        Beyond average position, how consistently does a player occupy their role's spatial zone?
    \end{block}
    \textbf{Definition:} Standard deviation of Euclidean distance from the role center.
    
    \begin{columns}
        \begin{column}{0.5\textwidth}
            \begin{itemize}
                \item \textbf{Fixed Roles:} CBs show low Stationarity.
                \item \textbf{Roaming Roles:} Wingers/Midfielders show high Stationarity.
                \item Validates tactical roles beyond simple coordinates.
            \end{itemize}
        \end{column}
        \begin{column}{0.5\textwidth}
            \centering
            \includegraphics[width=\linewidth]{../figures/Experiments_Player_Stationarity_Phase1_1925299.png}
            \captionof{figure}{Player stationarity during Phase 1 of match "1925299" (A-League, SkillCorner) separated by phase of play: attacking (red) vs. defending (blue).}
        \end{column}
    \end{columns}
\end{frame}


%%%%%%%%%%%%%%%%%%%%%%%
% EVENT DATA
%%%%%%%%%%%%%%%%%%%%%%%
\subsection{Extension 3: Event Data Generalization}
\begin{frame}{Validation 1: Event Data Formations}
    \frametitle{Offensive vs. Defensive Structures (FC Barcelona)}
    \begin{block}{Challenge}
        Sparse event data lacks continuous player trajectories. Does the SoccerCPD framework still reveal tactical shapes?
    \end{block}

    \begin{columns}
        \begin{column}{0.5\textwidth}
            \begin{itemize}
                \item Despite low data density, distinct shapes emerge when aggregated over time.
                \item \textbf{Defense:} High density in some zones (winning back possession).
                \item \textbf{Attack:} More spread out, reflects offensive maneuvers.
            \end{itemize}
        \end{column}
        \begin{column}{0.5\textwidth}
            \centering
            \includegraphics[width=0.8\linewidth]{../figures/Experiments_Barca_Formations_By_Phase.png} 
            \captionof{figure}{Detected formations for FC Barcelona using StatsBomb data separated by different tactical phases within the match.}
        \end{column}
    \end{columns}
\end{frame}

\begin{frame}{Validation 2: Event Data Stationarity}
    \frametitle{Quantifying Fluidity in Event Data}
    \begin{block}{Challenge}
        Sparse event data may not capture fine-grained movement.
    \end{block}
    \vspace{0.2cm}
    \begin{columns}
        \begin{column}{0.5\textwidth}
            \textbf{Consistency:} The "Fixed" vs. "Fluid" distinction persists.
            
            \textbf{Phase Analysis (Barcelona):} 
                \begin{itemize}
                    \item Similar average stationarity across phases.
                    \item But different players show role fluidity shifts between phases.
                \end{itemize}
        \end{column}

        \begin{column}{0.5\textwidth}
            \centering
            \includegraphics[width=\linewidth]{../figures/Experiments_Barca_Stationarity_Phase1.png} 
            \captionof{figure}{Player stationarity during Phase 1 of FC Barcelona match (StatsBomb).}
        \end{column}
    \end{columns}
\end{frame}